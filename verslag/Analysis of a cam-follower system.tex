\documentclass[a4paper]{article}

\usepackage[utf8]{inputenc}
\usepackage[english]{babel}
\usepackage{hyperref}
\usepackage{graphicx}
\usepackage{amsmath,amssymb,amsthm}
\usepackage{siunitx}
\usepackage{xcolor}
\usepackage{multicol}
\usepackage{caption}
\usepackage{appendix}
\usepackage{pdfpages}
\usepackage{fixltx2e}
\usepackage[version=4]{mhchem}
\usepackage{url}
\usepackage{subcaption}
\usepackage{subfig}
\usepackage[justification=centering]{caption}
\usepackage[thinc]{esdiff}
\usepackage[bottom]{footmisc}
\usepackage{gensymb}
\usepackage[htt]{hyphenat}
\usepackage{float}
\usepackage{bm}
\usepackage{a4wide}


\date{\today}
\author{Thomas Brzeski \and  Victor Dejans}
\title{Analysis of a cam-follower system}


\begin{document}
	

\maketitle

\section*{Introduction}

In the context of the subject \textit{Beweging en trillingen (H01N0A)} this report treats the design and analysis of a cam-follower system. It is based on provided data from \texttt{num\_data.html} (number 46) which are also listed below:
\\

\fbox{
	\parbox{.9\textwidth}{
	
The cam must be able to accomplish the lift below:
\\


from 0 to 60 degrees: +20 mm \\
from 60° to 120 degrees: +15 mm \\
from 150° to 290 degrees : -35 mm \\


The equivalent mass and damping constant of the follower (and its parts) are respectively estimated at 20 kg and 0,054, while the mechanism must exercise the static forces below:\\

from 60° to 110 degrees: a linear increasing pressure force from 0 N to 150 N.\\
from 110 ° to 160 degrees: a constant pressure force of 250 N.\\
from 160 ° to 250 degrees: a constant pulling force of 230 N.\\

The requested cycle time for the operation performed by the follower is 1 second.}}
\\
\\



The first section defines the motion law of the cam-follower system. In a second section, the geometric parameters are determined, such as the radiuses of the cam and follower and the excentricity. The third section is about the rigid body forces and calculates the optimal spring characteristics and the power to drive the cam. The last section gives an overview of the dynamics of the follower.

The calculations for this assignment are made using MATLAB. The used code is attached to this report.

\clearpage
\tableofcontents

\section{Defining the motion law}

The motion law gives the displacement \(S(\theta)\) for each angle. The MATLAB function \texttt{matcam.m} was used to construct this motion law, which is a series of cycloidal and semi-cycloidal segments as defined in the manual \cite{cursus} in chapter 7 on slide 44.

When choosing the segments, one must take two things into account:
\begin{itemize}
	\item The velocity function \(S'(\theta)\) must be as continuous as possible.
	\item The acceleration \(S''(\theta)\) must be as low as possible.
\end{itemize}


\begin{figure}
	\centering
	\includegraphics[width=.7\textwidth]{hefwet.png}
	\caption{The motion law \(S(\theta)\) and its first and second derivatives in function of the cam angle \(\theta\). These plots were created with \texttt{matcam.m}}
	\label{fig:hefwet}
	
\end{figure}

The chosen segment sequence for this system consists of a C1, a C2 and a C6 segment and some dwells (segment with no rise or decline, which leads to a constant displacement).

\begin{itemize}

\item The first segment is a C1 half-cycloid which rises from 0 to +20 millimeters between 0 and 60 degrees. 

\item The second one is a C2 half-cycloid that rises from +20 to +35 millimeters between 60 and 105 degrees. By choosing to end this segment at 105 degrees, the first derivative of S is perfectly continuous in \(\theta=60\degree\).

\item After this second half-cycloid comes a dwell at +35 millimeters between 105 and 150 degrees. This dwell forms a perfectly continuous funtion with the C2 half-cycloid before it and the decreasing C6 cycloid behind it.

\item The last cycloidal part is a C6 segment that declines from +35 to 0 millimeters between 150 and 290 degrees.

\item To connect the C6 cycloid back with the C1 segment, a dwell at 0 millimeters is used between 290 and 360 degrees.

\end{itemize}

A summary of these segments is given in table~\ref{tab:motionlaw}. A graphical representation of this motion law and of its first and second derivatives is shown in figure~\ref{fig:hefwet}. On this plot one can see that the velocity is perfectly continuous and that the absolute value of the acceleration does not exceed \(0,025~\si{mm/degree^2}\).

\begin{table}[h]
	\centering
	\resizebox{\textwidth}{!}{\begin{tabular}{c|cccc}
		Segment & \(\beta_{start}\) (degrees) & \(\beta_{end}\) (degrees) & Cycloid type & Relative displacement (mm) \\
		\hline
		1 & 0 & 60 & C1 & +20 \\
		2 & 60 & 105 & C2 & +15\\
		3 & 105 & 150 & dwell & 0\\
		4 & 150 & 290 & C6 & -35\\
		5 & 290 & 360 & dwell & 0\\
	\end{tabular}}
	\caption{Results of the use of the Kloomok and Muffley diagram for each segment with a lift \(L\neq0\).}
	\label{tab:motionlaw}
\end{table}

\section{Determining the geometry of the follower}
\label{sec:2}

Some geometric characteristics of the follower must be chosen wisely to optimize the cam's operation. The choice of these parameters is the subject of this section.

The radiuses of the base circle of the cam and of the follower's roller must be chosen so that there is no undercutting and that the pressure angle never exceeds 30 degrees. Then the excentricity can help to make the maximal pressure angle even smaller.

\subsection{Sum of the radiuses of base circle and follower, \(R_0\)}

First, the pitch circle radius \(R_0\) is calculated. This is the sum of the base circle radius \(R_{base}\) and the follower radius \(R_r\), the two variables that need to be found in this subsection. 

This can be done by looking at the pressure angle. This is the angle between the direction in which the follower translates and the normal on the cam's surface. It gives an indication of how the transmitted force between cam and follower is converted. When the pressure angle is small, most of this force will be used for the follower's motion. When the pressure angle is more right (\(\approx90\degree\)), most of it will be converted in frictional forces. The latter case must of course be avoided. A good rule of thumb states that the pressure angle may never exceed 30 degrees:

\begin{equation}
	\alpha_{max} < 30 \degree 
\end{equation}

The Kloomok and Muffley nomogram on slide 33 of chapter 8 in the manual~\cite{cursus} gives the relation between the pressure angle \(\alpha\) and the ratio of the displacement to the pitch circle radius \(L/R_0\). For each segment of the motion law, a line is drawn in this nomogram between the point of \(\alpha=30 \degree\) (the maximal allowed value for \(\alpha\)) and the point of the \(\beta\) which represents the size in degrees of the considered segment. The intersection of this drawn line and the horizontal axis in the nomogram gives the \(L/R_0\) ratio for that cycloidal segment. An example of how to use the nomogram is given in figure~\ref{fig:nomogram1}.

The results for each segment are listed in table~\ref{tab:nomogram}. Because the displacement \(L\) is known for each segment, the pitch circle radius \(R_0\) can be calculated for each segment. The highest value for \(R_0\) is 60 millimeters. By choosing this value or higher, the pressure angle will surely not exceed 30 degrees. This will still be optimized by choosing an excentricity, so 60 millimeters is good enough for now.

\begin{figure}
	\centering
	\includegraphics[width=.66\textwidth]{nomogram1.png}
	\caption{Nomogram of Kloomok and Muffley used to determine the pitch circle radius \(R_0\).}
	\label{fig:nomogram1}
\end{figure}

\begin{table}
	\centering
	\begin{tabular}{c|cccc}
		Segment & \(\beta\) (degrees) & \(L\) (mm) & \(L/R_0\) & \(R_0\) (mm) \\
		\hline
		1 & 60 & 20 & 0,33 & 60 \\
		2 & 45 & 15 & 0,25 & 60\\
		4 & 35 & 35 & 0,90 & 39\\
	\end{tabular}
	\caption{Results of the use of the Kloomok and Muffley diagram for each segment with a lift \(L\neq0\).}
	\label{tab:nomogram}
\end{table}

\subsection{Radius of the base circle, \(R_{base}\) and radius of the follower, \(R_r\)}

The conclusion of the previous subsection was that \(R_{base}+R_r=60~mm\). Now the base circle radius and the follower radius must be determined seperately. This can be done by stating that to avoid undercutting, the radius of curvature of the cam's surface must always be greater than the follower radius:

\begin{equation}
	\rho _{min} > R_r
\end{equation}

To calculate this minimal radius of curvature, another nomogram of Kloomok and Muffley is used. It is generated by the MATLAB function \texttt{gen\_fig\_Kloomok\_Muffley.m}. This program makes plots of the \(\rho_{min}\) of each segment given the pitch radius \(R_0\), the lift and the cycloid type. These plots are listed in figure~\ref{fig:nomogram2}.

From these plots, one can see that the smallest value for \(\rho_{min}\) is 60 millimeters, which means that the follower radius must be smaller than 60 millimeters. This is just an upper boundary, so it can be a lot smaller than that. We have chosen to make \textbf{\(R_r=10~mm\)}, so that the base circle radius is \textbf{\(R_{base}=50~mm\)}. It is desirable that the follower radius is smaller than the base circle radius to make the motion smoother.

Plots of the pressure angle and the radius of curvature with these radiuses can be seen in figure~\ref{fig:geozonder}.

\begin{figure}
	\centering
	
	\begin{subfigure}{.7\textwidth}
		\centering
		\includegraphics[width=\textwidth]{preszonder.png}
		\caption{Pressure angle in function of cam angle. For smooth motion, this angle must never exceed 30 degrees.}
		\label{fig:preszonder}
	\end{subfigure}
	\hfill
	\begin{subfigure}{.7\textwidth}
		\centering
		\includegraphics[width=\textwidth]{radzonder.png}
		\caption{Radius of curvature in function of cam angle. To avoid undercutting, the follower radius \(R_r\) must always be smaller than the pitch radius of curvature (blue line).}
		\label{fig:radzonder}
	\end{subfigure}
	
	\caption{Pressure angle \(\alpha\) and radius of curvature \(\rho\) in function of cam angle for the case of a centric follower with \(R_r=10~mm\) and \(R_{base}=50~mm\).}
	\label{fig:geozonder}
	
\end{figure}

\begin{figure}
	\centering
	
	\includegraphics[width=\textwidth]{nomogram2.png}
	
	
	\caption{Nomograms of Kloomok and Muffley for determining the \(\rho_{min}\).}
	\label{fig:nomogram2}
\end{figure}

\subsection{Excentricity \(e\)}

In the previous subsections, the follower was assumed to be in line with the center of the cam. The follower can however also be placed excentrically relative to the cam. This can help to reduce the pressure angle.

When choosing an excentricity \(e\) different from zero, the curve of the pressure angle in function of the cam angle, which is plotted in figure~\ref{fig:preszonder}, moves up or down. The intention is to find an excentricity so that the maximal value and the minimal value of the pressure angle lie symmetrically relative to the horizontal axis. In that case, the maximal value for \(|\alpha|\) is minimized.

Finding the ideal excentricity was done by trial and error in \texttt{matcam.m}. For \textbf{\(e=4,5~mm\)} the pressure angle is optimal. It can be seen in figure~\ref{fig:presmet}. The new radius of curvature can be seen in figure~\ref{fig:radmet}.

\begin{figure}
	\centering
	
	\begin{subfigure}{.7\textwidth}
		\centering
		\includegraphics[width=\textwidth]{presmet.png}
		\caption{Pressure angle in function of cam angle. For smooth motion, this angle must never exceed 30 degrees.}
		\label{fig:presmet}
	\end{subfigure}
	\hfill
	\begin{subfigure}{.7\textwidth}
		\centering
		\includegraphics[width=\textwidth]{radmet.png}
		\caption{Radius of curvature in function of cam angle. To avoid undercutting, the follower radius \(R_r\) must always be smaller than the pitch radius of curvature (blue line).}
		\label{fig:radmet}
	\end{subfigure}
	
	\caption{Pressure angle \(\alpha\) and radius of curvature \(\rho\) in function of cam angle for the case of an excentric follower with \(R_r=10~mm\), \(R_{base}=50~mm\) and \(e=4,5~mm\).}
	\label{fig:geomet}
	
\end{figure}

\section{Verifying the rigid body forces}

This section treats the calculation and verification of the rigid body forces. This means that the follower is considered to be rigid and that the driving speed is constant and equal to \(\omega_{nom}\).

\subsection{Sizing the spring}

The rigid follower is attached to a spring so that the contact between the cam and the follower can always be assured.

This means that the total normal force exercised by the cam on the follower, must always be positive. Otherwise the cam would apply a traction force on the follower, which is impossible in this configuration. This normal force can be plotted by \texttt{matcam.m}. Figure~\ref{fig:Nzonderveer} shows this plot when no spring is used. 

\begin{figure}
	\centering
	\includegraphics[width=.7\textwidth]{Nzonderveer.png}
	\caption{Normal force between cam and follower in function of the cam angle for a follower without spring.}
	\label{fig:Nzonderveer}
\end{figure}

\subsubsection{Spring with nominal driving velocity}

To get the normal force above zero, a spring with a certain spring constant \(k\) and a preload \(F_{v0}\) (a constant force that is exercised by the spring) are applied. On slide 49 of chapter 8 of the manual~\cite{cursus}, the analytical method to find \(k\) and \(F_{v0}\) is given:

\begin{equation}
	F_{v0} + k\cdot S(\theta) \geq -F_{func}(\theta) -m\cdot\omega^2\cdot\diff[2]{S(\theta)}{\theta}
\end{equation}

\begin{equation}
	k = \max_\theta\left(\frac{-F_{func}(\theta)-F_{v0}-m\cdot\omega^2\cdot\diff[2]{S(\theta)}{\theta}}{S(\theta)}\right)
\end{equation}

where \(F_{func}\) is the external force applied on the cam as defined in the assignment.

Increasing \(k\) or \(F_{v0}\) helps to increase the normal force. However, when the spring constant is too high, the maximal normal force will be too high to guarantee good motion between the cam and the follower. When the preload is too high, a very large spring will be required.

Thus, a compromise must be found between preload and spring constant. By trial and error in \texttt{matcam.m}, we decided to make \(k = 7~N/m\) and \(F_{v0} = 230~N\). The normal force between cam and angle can be seen in figure~\ref{fig:Nmetveer}.

\begin{figure}
	\centering
	\includegraphics[width=.7\textwidth]{Nmetveer.png}
	\caption{Normal force between cam and follower in function of the cam angle for a follower with spring with \(k=7~N/m\) and \(F_{v0}=230~N\) at nominal velocity.}
	\label{fig:Nmetveer}
\end{figure}

\subsubsection{Spring with deviating driving velocity}

The previous subsection treats the design of a spring when the cam's angular velocity is \(\omega_{nom}=60~degrees/s\). In this subsection, the velocity is doubled, so 

\begin{equation}
	\(\omega=2\cdot\omega_{nom}=120~degrees/s\). 
\end{equation}


The plot of the total normal force and its components for this case is shown in figure~\ref{fig:Ndubbelw}. The external force and the spring force are independent from the angular velocity and are thus unchanged. The inertial force however ("acc" in figure~\ref{fig:Ndubbelw}), do change. This force is given by~\cite{cursus}:

\begin{equation}
	\begin{split}
	F_{inertial} &= m\cdot a \\
	&=m\cdot\left(\frac{L}{\beta^2}\cdot\omega^2\cdot s''(\tau)\right)
	\end{split}
\end{equation}

This means that this force is proportional to the square of the angular velocity. When the velocity is doubled, the inertial force will be multiplied by four.

The total normal force is still greater than zero for all cam angles, so the spring does not need to be resized for this velocity.

\begin{figure}
	\centering
	\includegraphics[width=.7\textwidth]{Ndubbelw.png}
	\caption{Normal force between cam and follower in function of the cam angle for a follower with spring with \(k=7~N/m\) and \(F_{v0}=230~N\) at \(\omega=2\cdot\omega_{nom}\).}
	\label{fig:Ndubbelw}
\end{figure}


\subsection{Instantaneous power}
\label{sec:power}

The power to drive the cam is given for a non-excentric follower \cite{vermogen}:

\begin{equation}
	\begin{split}
	P(\theta) & = \vec{N}(\theta)\cdot\vec{v}(\theta) \\
	&=N(\theta)\cdot sin(\alpha)\cdot R(\theta)\cdot \omega
	\end{split}
\end{equation}

For the case where there is an excentrity \(e\neq0\), the power is calculated as follows:

\begin{equation}
	\begin{split}
	P(\theta) & = \vec{N}(\theta)\cdot\vec{v}(\theta) \\
	&=N(\theta)\cdot cos(\alpha)\cdot f'(\theta)\cdot\omega
	\end{split}
	\label{eq:verm_exc1}
\end{equation}

The pressure angle for a cam with an excentric follower is given in slide 31 of chapter 8 of the manual~\cite{cursus}: FIGUURFIGUURGIFURGIFURIFUFRIFUUUUUFFF

\begin{equation}
	\begin{split}
	\alpha& = arctan\left(\frac{f'(\theta)-e}{\sqrt{R_0^2-e^2}+f(\theta)}\right)\\
	&=arctan\left(\frac{f'(\theta)-e}{\sqrt{R(\theta)^2-e^2}}\right)
	\end{split}
	\label{eq:verm_exc2}
\end{equation}

And by using equation~\ref{eq:verm_exc2} in equation~\ref{eq:verm_exc1}, the power can be written as:

\begin{equation}
	\begin{split}
	P(\theta) =&~ N(\theta)\cdot cos(\alpha)\cdot \Big(tan(\alpha)\cdot\sqrt{R(\theta)^2-e^2}+e\Big)\cdot\omega\\
	=&~N(\theta)\cdot sin(\alpha)\cdot\sqrt{R(\theta)^2-e^2}\cdot\omega  +~N(\theta)\cdot cos(\alpha)\cdot e \cdot\omega
	\end{split}
\end{equation}

The power for both the case without excentricity and the case with excentricity (\(e=4,5~cm\) as determined in section~\ref{sec:2}) is calculated in the MATLAB function \texttt{power\_cam.m}. The plots of the powers in function of \(\theta\) can be found in figure~\ref{fig:powerplot}. Figure~\ref{diffpower} shows the difference between the two power plots. This difference is of an order of magnitude of \(10^{-14}\) and is due to the machine precision of MATLAB. The power for the cam with a centric follower is thus equal to the power for the cam with an excentric follower.

The horizontal lines in the plots in figure~\ref{fig:powerplot} represent the average value of the instantaneous power. For this system, the mean power is positive which means that the system consumes energy. The mechanism is thus a motor and not a generator. Energy is transmitted from the cam to the follower. It is dissipated in the form of motion of the follower and friction between cam and follower.

\begin{figure}
	\centering
	\includegraphics[width=\textwidth]{powerplot.png}
	\caption{The instantaneous power \(P\) and average to drive the cam in function of \(\theta\) for a centric and an excentric follower (in blue). The mean value of the power \(P_{mean}\) (in red).}
	\label{fig:powerplot}
	
\end{figure}

\begin{figure}
	\centering
	\includegraphics[width=\textwidth]{diffpower.png}
	\caption{Difference between the power with and without excentricity in function of \(\theta\).}
	\label{diffpower}
\end{figure}

\subsection{Designing a flywheel}

This section treats the dimensioning of a flywheel that will keep speed variations in the cam shaft under control. The exact calculation of the moment of inertia of the flywheel is done in the first subsubsection. It is based on the theory described in slides 15 to 22 of chapter 4 in the manual~\cite{cursus}. In a second subsubsection, the dimensions are estimated based on the plot of the torques.

\subsubsection{Exact calculation of the flywheel's dimensions}

First the instantaneous en average torque is calculated. This can easily be done by dividing the instantaneous and average power (that are already calculated in section~\ref{sec:power}) by the nominal angular velocity:
\begin{subequations}
\begin{equation}
	M(\theta)=\frac{P(\theta)}{\omega_{nom}}
\end{equation}
\begin{equation}
	\overline{M}=\frac{\overline{P}}{\omega_{nom}}
\end{equation}
\end{subequations}

A plot of the torques in function of the cam angle is shown in figure~\ref{fig:torque}. 

The difference between the instantaneous and average torque causes the speed variations that the flywheel will suppress. By integrating this difference, the work surplus \(A(\theta)\) is found:

\begin{equation}
	A(\theta)=\int_{0}^{\theta}(M(\theta)-\overline{M})d\theta
\end{equation}

This \(A(\theta)\) is plotted in figure~\ref{fig:work}. By integrating it between the angle with minimal crank speed and the angle with maximal crank speed, the maximum work surplus is found:

\begin{equation}
	\begin{split}
	A_{max}&=\int_{\theta_{min}}^{\theta_{max}}(M(\theta)-\overline{M})d\theta\\
	&=\int_{0,24}^{1,73}(M(\theta)-1,20~Nm)d\theta\\
	&=11,02~J
	\end{split}
	\label{eq:amax}
\end{equation}

This maximum work surplus is then used to calculate the moment of inertia as follows:

\begin{equation}
	I = \frac{A_{max}}{K\cdot\omega_{nom}^2}
\end{equation}

where K is the fluctuation coefficient which is defined as:

\begin{equation}
	K = \frac{\omega_{max}-\omega_{min}}{\omega_{nom}}
\end{equation}

These maximal and angular velocities are respectively \(0,95\cdot\omega_{nom}\) and \(1,05\cdot\omega_{nom}\), because of the tolerance which was defined in the assignment. After filling this all in, the moment of inertia is \(I=2,79~kg\cdot m^2\).

\begin{figure}
	\centering
	\includegraphics[width=.9\textwidth]{torque.png}
	\caption{Instantaneous torque \(M(\theta)\) (blue) and average torque \(\overline{M}\) (red) in function of the cam angle. The vertical lines show where the angles with minimal (left) and maximal (right) crank speed.}
	\label{fig:torque}
\end{figure}
\begin{figure}
	\centering
	\includegraphics[width=.9\textwidth]{work.png}
	\caption{Instantaneous work surplus \(A(\theta)\).}
	\label{fig:work}
\end{figure}

HOE GROOT PRECIES???????

\subsubsection{Estimate of the flywheel's dimensions}

This calculation can also be estimated by looking at the plot of the torques in figure~\ref{fig:torque}. The \(A_{max}\) is then the area between the blue curve and the red line and between the left and right vertical lines. This can be estimated as the area of a triangle with a base of 80 or 90 degrees, or 1,5 radians, and a height of 13~Nm. The area (and also maximum work surplus) is then:
\begin{equation}
	A_{max}\approx\frac{1,5*13~Nm}{2}\approx 10~J
\end{equation} 
This is a good estimate for the 11,02~J from equation~\ref{eq:amax}. The moment of inertia of the flywheel can then be approximated as:
\begin{equation}
	\begin{split}
	I &\approx\frac{10~J}{(\frac{1,05\cdot \omega_{nom}-0,95\cdot\omega_{nom}}{\omega_{nom}})\cdot\omega_{nom}^2}\\
	 &\approx \frac{10~J}{0,10\cdot(2\pi~/s)^2}\\
	 &\approx \frac{10~\frac{kg\cdot m^2}{s^2}}{0,10\cdot 6,3~/s \cdot 6,3~/s}\\
	 &\approx \frac{100}{40}~kg\cdot m^2\\
	 &\approx 2,5~kg\cdot m^2\\
	\end{split}
\end{equation}
This is a good estimate for the exact calculation of the moment of inertia.

\subsection{Startup of the motor}

Because the power is positive, the cam has to be driven by a motor. When the motor is running, it has the specifications that are described in the previous subsections. When the motor is started up, the behaviour is different. The motor will need to perform an extra torque because of the inertia of the cam and the flywheel which are at rest. This extra torque is given by:

\begin{equation}
	M_{inertia} = I \cdot \alpha
\end{equation}

where \(\alpha\) is the angular acceleration of the cam when the motor is starting up. 

When using an electric motor, the current will peak during startup. This has to be avoided because it can damage the motor and can cause troubles in the local electric grid. To solve this problem, an upstream transformer is placed in the motor that reduces the voltage during startup and is then detached during the working regime~\cite{energie}.



\section{Dynamics of the follower}

\subsection{Single-rise analysis}

\subsection{Multi-rise analysis}

\section*{Conclusion}

We can conclude that

\bibliographystyle{plain}
\bibliography{nokken}

\end{document}